% Options for packages loaded elsewhere
% Options for packages loaded elsewhere
\PassOptionsToPackage{unicode}{hyperref}
\PassOptionsToPackage{hyphens}{url}
\PassOptionsToPackage{dvipsnames,svgnames,x11names}{xcolor}
%
\documentclass[
  letterpaper,
  DIV=11,
  numbers=noendperiod]{scrartcl}
\usepackage{xcolor}
\usepackage{amsmath,amssymb}
\setcounter{secnumdepth}{-\maxdimen} % remove section numbering
\usepackage{iftex}
\ifPDFTeX
  \usepackage[T1]{fontenc}
  \usepackage[utf8]{inputenc}
  \usepackage{textcomp} % provide euro and other symbols
\else % if luatex or xetex
  \usepackage{unicode-math} % this also loads fontspec
  \defaultfontfeatures{Scale=MatchLowercase}
  \defaultfontfeatures[\rmfamily]{Ligatures=TeX,Scale=1}
\fi
\usepackage{lmodern}
\ifPDFTeX\else
  % xetex/luatex font selection
\fi
% Use upquote if available, for straight quotes in verbatim environments
\IfFileExists{upquote.sty}{\usepackage{upquote}}{}
\IfFileExists{microtype.sty}{% use microtype if available
  \usepackage[]{microtype}
  \UseMicrotypeSet[protrusion]{basicmath} % disable protrusion for tt fonts
}{}
\makeatletter
\@ifundefined{KOMAClassName}{% if non-KOMA class
  \IfFileExists{parskip.sty}{%
    \usepackage{parskip}
  }{% else
    \setlength{\parindent}{0pt}
    \setlength{\parskip}{6pt plus 2pt minus 1pt}}
}{% if KOMA class
  \KOMAoptions{parskip=half}}
\makeatother
% Make \paragraph and \subparagraph free-standing
\makeatletter
\ifx\paragraph\undefined\else
  \let\oldparagraph\paragraph
  \renewcommand{\paragraph}{
    \@ifstar
      \xxxParagraphStar
      \xxxParagraphNoStar
  }
  \newcommand{\xxxParagraphStar}[1]{\oldparagraph*{#1}\mbox{}}
  \newcommand{\xxxParagraphNoStar}[1]{\oldparagraph{#1}\mbox{}}
\fi
\ifx\subparagraph\undefined\else
  \let\oldsubparagraph\subparagraph
  \renewcommand{\subparagraph}{
    \@ifstar
      \xxxSubParagraphStar
      \xxxSubParagraphNoStar
  }
  \newcommand{\xxxSubParagraphStar}[1]{\oldsubparagraph*{#1}\mbox{}}
  \newcommand{\xxxSubParagraphNoStar}[1]{\oldsubparagraph{#1}\mbox{}}
\fi
\makeatother


\usepackage{longtable,booktabs,array}
\usepackage{calc} % for calculating minipage widths
% Correct order of tables after \paragraph or \subparagraph
\usepackage{etoolbox}
\makeatletter
\patchcmd\longtable{\par}{\if@noskipsec\mbox{}\fi\par}{}{}
\makeatother
% Allow footnotes in longtable head/foot
\IfFileExists{footnotehyper.sty}{\usepackage{footnotehyper}}{\usepackage{footnote}}
\makesavenoteenv{longtable}
\usepackage{graphicx}
\makeatletter
\newsavebox\pandoc@box
\newcommand*\pandocbounded[1]{% scales image to fit in text height/width
  \sbox\pandoc@box{#1}%
  \Gscale@div\@tempa{\textheight}{\dimexpr\ht\pandoc@box+\dp\pandoc@box\relax}%
  \Gscale@div\@tempb{\linewidth}{\wd\pandoc@box}%
  \ifdim\@tempb\p@<\@tempa\p@\let\@tempa\@tempb\fi% select the smaller of both
  \ifdim\@tempa\p@<\p@\scalebox{\@tempa}{\usebox\pandoc@box}%
  \else\usebox{\pandoc@box}%
  \fi%
}
% Set default figure placement to htbp
\def\fps@figure{htbp}
\makeatother





\setlength{\emergencystretch}{3em} % prevent overfull lines

\providecommand{\tightlist}{%
  \setlength{\itemsep}{0pt}\setlength{\parskip}{0pt}}



 


\KOMAoption{captions}{tableheading}
\makeatletter
\@ifpackageloaded{caption}{}{\usepackage{caption}}
\AtBeginDocument{%
\ifdefined\contentsname
  \renewcommand*\contentsname{Table of contents}
\else
  \newcommand\contentsname{Table of contents}
\fi
\ifdefined\listfigurename
  \renewcommand*\listfigurename{List of Figures}
\else
  \newcommand\listfigurename{List of Figures}
\fi
\ifdefined\listtablename
  \renewcommand*\listtablename{List of Tables}
\else
  \newcommand\listtablename{List of Tables}
\fi
\ifdefined\figurename
  \renewcommand*\figurename{Figure}
\else
  \newcommand\figurename{Figure}
\fi
\ifdefined\tablename
  \renewcommand*\tablename{Table}
\else
  \newcommand\tablename{Table}
\fi
}
\@ifpackageloaded{float}{}{\usepackage{float}}
\floatstyle{ruled}
\@ifundefined{c@chapter}{\newfloat{codelisting}{h}{lop}}{\newfloat{codelisting}{h}{lop}[chapter]}
\floatname{codelisting}{Listing}
\newcommand*\listoflistings{\listof{codelisting}{List of Listings}}
\makeatother
\makeatletter
\makeatother
\makeatletter
\@ifpackageloaded{caption}{}{\usepackage{caption}}
\@ifpackageloaded{subcaption}{}{\usepackage{subcaption}}
\makeatother
\usepackage{bookmark}
\IfFileExists{xurl.sty}{\usepackage{xurl}}{} % add URL line breaks if available
\urlstyle{same}
\hypersetup{
  pdftitle={Esophagectomy},
  colorlinks=true,
  linkcolor={blue},
  filecolor={Maroon},
  citecolor={Blue},
  urlcolor={Blue},
  pdfcreator={LaTeX via pandoc}}


\title{Esophagectomy}
\author{}
\date{}
\begin{document}
\maketitle


\subsection{Introduction}\label{introduction}

I'm Dr Jonathan Salo, a GI Cancer Surgeon in Charlotte, North Carolina.

In this video, you will learn about

\begin{itemize}
\tightlist
\item
  Different kinds of surgery for esophageal cancer
\item
  Risks of surgery
\item
  How you can reduce the risk of surgery
\end{itemize}

In another video, we'll talk about how to choose a hospital and surgeon
for your esophagectomy.

\begin{center}\rule{0.5\linewidth}{0.5pt}\end{center}

Surgery for esophageal cancer is generally performed for three different
situations:

\begin{itemize}
\tightlist
\item
  Superficial Tumors (T1) that can't be completely removed by endoscopy
\item
  Localized Tumors (T2N0)
\item
  Locally Advanced Tumors (T3 or N+) after the completion of
  preoperative therapy. Preoperative therapy is generally some
  combination of chemotherapy and radiation.
\end{itemize}

If you haven't seen it already, this may be a good time to view the
Esophageal Cancer Treatment Options video. There's a link above and a
link in the description below.

\href{eso_rx_options.qmd}{Esophageal Cancer Treatment Options}

\begin{center}\rule{0.5\linewidth}{0.5pt}\end{center}

\subsection{Goals of Esophagectomy}\label{goals-of-esophagectomy}

\begin{itemize}
\tightlist
\item
  Remove tumor from esophagus
\item
  Remove surrounding lymph nodes
\item
  Create a new esophagus
\end{itemize}

\pandocbounded{\includegraphics[keepaspectratio]{images/Eso_Resection1_ai.png}}

\begin{center}\rule{0.5\linewidth}{0.5pt}\end{center}

\subsection{Resection}\label{resection}

The \emph{Ivor Lewis} esophagectomy, shown here, removes the lower 2/3
of the esophagus, the tumor, and the surrounding lymph nodes.

\pandocbounded{\includegraphics[keepaspectratio]{images/Eso_Resection2_ai.png}}

\begin{center}\rule{0.5\linewidth}{0.5pt}\end{center}

\subsection{Reconstruction}\label{reconstruction}

A new esophagus is created from the stomach in the abdomen by fashioning
it into a tube.

\pandocbounded{\includegraphics[keepaspectratio]{images/Eso_Resection3_ai.png}}

\begin{center}\rule{0.5\linewidth}{0.5pt}\end{center}

\subsection{Ivor Lewis esophagectomy}\label{ivor-lewis-esophagectomy}

The new esophagus is now brought up into the chest. A new connection is
made between the esophagus and the stomach, called an
\emph{anastomosis}.

\pandocbounded{\includegraphics[keepaspectratio]{images/Eso_IvorLewis_Anastomosis.png}}

\begin{center}\rule{0.5\linewidth}{0.5pt}\end{center}

\subsection{Open Esophagectomy}\label{open-esophagectomy}

Open esophagectomy uses conventional incisions in the abdomen and the
right chest. An incision is made between the ribs on the right side, and
an abdominal incision made from the breast bone to below the belly
button. This is a well-established surgical approach which has been used
for the past 75 years.

\pandocbounded{\includegraphics[keepaspectratio]{images/eso_IvorLewis_ai.png}}

\begin{center}\rule{0.5\linewidth}{0.5pt}\end{center}

\subsection{Minimally-invasive Ivor
Lewis}\label{minimally-invasive-ivor-lewis}

Mininally-invasive esophagectomy uses small incisions in the abdomen and
chest. A surgical telescope and special instruments are used to perform
the operations. This operation is a more recent innovation and can be
used in many cases instead of an open approach.

The smaller incisions mean faster recovery and less discomfort

\pandocbounded{\includegraphics[keepaspectratio]{images/MIE_IvorLewisArtboard 2@4x.png}}

\begin{center}\rule{0.5\linewidth}{0.5pt}\end{center}

\subsection{Minimally-invasive Ivor
Lewis}\label{minimally-invasive-ivor-lewis-1}

We have found this is the best option for most of our patients. In some
cases, an open approach is still necessary.

\pandocbounded{\includegraphics[keepaspectratio]{images/MIE_IvorLewisArtboard_2.png}}

\begin{center}\rule{0.5\linewidth}{0.5pt}\end{center}

\subsection{Total Esophagectomy}\label{total-esophagectomy}

For patients with tumors in the upper esophagus, we need to remove more
of the esophagus

\pandocbounded{\includegraphics[keepaspectratio]{images/Eso_ProxTumorArtboard_2.png}}

\begin{center}\rule{0.5\linewidth}{0.5pt}\end{center}

\subsection{Total Esophagectomy}\label{total-esophagectomy-1}

For those patients, we need to remove the whole esophagus

\pandocbounded{\includegraphics[keepaspectratio]{images/Eso_ResectionTotalArtboard_2.png}}

\begin{center}\rule{0.5\linewidth}{0.5pt}\end{center}

\subsection{Minimally-invasive McKeown
Esophagectomy}\label{minimally-invasive-mckeown-esophagectomy}

In this case, a connection between the esophagus and the stomach is made
in the neck.

\pandocbounded{\includegraphics[keepaspectratio]{images/Eso_MIE_McKeownArtboard_2.png}}

\begin{center}\rule{0.5\linewidth}{0.5pt}\end{center}

\subsection{Transhiatal Esophagectomy}\label{transhiatal-esophagectomy}

Another option is a transhiatal esohagectomy, which avoids the need to
make incisions in the right chest. The operation is performed from the
abdomen and the right neck.

\pandocbounded{\includegraphics[keepaspectratio]{images/Eso_MIE_THEArtboard_2.png}}

\begin{center}\rule{0.5\linewidth}{0.5pt}\end{center}

When you meet with your surgeon, you will have an opportunity to discuss
your particular situation and their recommendation for surgery. Your
surgeon will recommend a surgical approach based upon you and your tumor
and their personal experience.

\begin{center}\rule{0.5\linewidth}{0.5pt}\end{center}

\subsection{Risks of Surgery}\label{risks-of-surgery}

An esophagectomy is a substantial operation, and in some cases there can
be postoperative complications. We're going to talk about two of these
complications and what you can do to reduce your risk of complications:

\begin{itemize}
\tightlist
\item
  Anastomotic leak
\item
  Pneumonia
\end{itemize}

\begin{center}\rule{0.5\linewidth}{0.5pt}\end{center}

\subsection{Anastomotic Leak}\label{anastomotic-leak}

The anastomosis is surigcal connection between the esophagus and the
stomach.

\pandocbounded{\includegraphics[keepaspectratio]{images/Eso_IL_AnastArtboard 2@4x.png}}

\begin{center}\rule{0.5\linewidth}{0.5pt}\end{center}

\subsection{Anastomotic Leak}\label{anastomotic-leak-1}

If anastomosis does not heal properly, this can cause a leakage of fluid
from the esophagus, called an anastomotic leak. If this happens, an
infection can occur in the mediastinum, which is the space near the
heart between the lungs.

\pandocbounded{\includegraphics[keepaspectratio]{images/Eso_LeakArtboard 2@4x.png}}

\begin{center}\rule{0.5\linewidth}{0.5pt}\end{center}

\subsection{Anastomotic Leak}\label{anastomotic-leak-2}

In some cases, the leak will heal on its own, but other cases may
require additional procedures or even surgery. The risk of leak depends
upon the operation performed but also depends upon the experience of the
surgeon. At the end of this video we have a link to a video about how to
choose a hospital and a surgeon, which talks further about the risks of
a leak.

\begin{center}\rule{0.5\linewidth}{0.5pt}\end{center}

\subsection{Pneumonia}\label{pneumonia}

Pneumonia is another complication which can occurs in about 10-15\% of
patients after esophagectomy. Pneumonia requires treatment with
antibiotics and frequently requires a longer hospitalization.

\pandocbounded{\includegraphics[keepaspectratio]{images/Eso_LungsArtboard 2@4x.png}}

\begin{center}\rule{0.5\linewidth}{0.5pt}\end{center}

\subsection{Preventing Pneumonia}\label{preventing-pneumonia}

In normal circumstances, secretions from the mouth and throat aren't
able to enter the lungs because we clear our throat and if secretions do
get into our airway, we tend to cough and keep those secretions out of
our lungs. This happens constantly without our thinking about it.

After esophagectomy, however, there is a tendency for secretions to
enter the airway, and if you can't clear them, there is a risk that
pneumonia will set in.

There are two important ways that pneumonia can be prevented:

\begin{itemize}
\tightlist
\item
  Deep breathing
\item
  Walking
\end{itemize}

\begin{center}\rule{0.5\linewidth}{0.5pt}\end{center}

\subsection{Deep breathing and
coughing}\label{deep-breathing-and-coughing}

After surgery, it's important to breathe deeply to help your lungs
recover after surgery. Deep breathing make the cough more effective and
helps clear secretions. After surgery, deep breathing and coughing can
be uncomfortable, so controlling your discomfort will be an important
part of your recovery.

\begin{center}\rule{0.5\linewidth}{0.5pt}\end{center}

\subsection{Walking}\label{walking}

Walking after surgery is also an important way to help your lungs
recover as well. When we walk, it's easier for our lungs to function,
and again, it makes the cough more frequently.

\begin{center}\rule{0.5\linewidth}{0.5pt}\end{center}

\subsection{Preventing Pneumonia}\label{preventing-pneumonia-1}

How can we prevent pneumonia? Believe it or not, I can tell who is more
likely to develop pneumonia after surgery when I first meet them and
shake their hand. Someone with a firm handshake has a lower risk of
pneumonia. We think this is because someone with a firm handshake has
good muscle tone, and someone with good muscle tone probably has good
function of the muscles between the ribs so that they have a nice strong
cough and can prevent pneumonia.

\subsection{Strength}\label{strength}

In our clinic, we actually measure out patient's strength with a
hand-held strength gauge called a dynamometer. Based upon these
measurements, we can identify patients who may be at risk of pneumonia.

\subsection{Patient Strength and Esophagectomy
Outcomes}\label{patient-strength-and-esophagectomy-outcomes}

About half of our patients have good strength, shown in green. A quarter
are have low strength, shown in red Another quarter are in the middle,
shown in yellow

\begin{center}\rule{0.5\linewidth}{0.5pt}\end{center}

\pandocbounded{\includegraphics[keepaspectratio]{eso_surgery_files/figure-pdf/pie-1.pdf}}

\begin{center}\rule{0.5\linewidth}{0.5pt}\end{center}

\subsection{Pneumonia}\label{pneumonia-1}

Overall, the risk of pneumonia is about 10\% in our patients who undergo
esophagectomy. 90\% of patients never experience pneumonia, but 10\%
will have pneumonia after surgery.

\begin{center}\rule{0.5\linewidth}{0.5pt}\end{center}

\pandocbounded{\includegraphics[keepaspectratio]{eso_surgery_files/figure-pdf/unnamed-chunk-1-1.pdf}}

\begin{center}\rule{0.5\linewidth}{0.5pt}\end{center}

\pandocbounded{\includegraphics[keepaspectratio]{eso_surgery_files/figure-pdf/pneumonia_figoverall-1.pdf}}

\begin{center}\rule{0.5\linewidth}{0.5pt}\end{center}

However the risk of pneumonia is not the same for everyone. Even though
the average risk is 10\%, the risk is much higher for our patients with
low muscle strength and much lower for patients with good muscle
strength.

For the half of our patients with good muscle strength, the risk of
pneumonia is about 5\%. On the other hand, the risk of pneumonia is 20\%
in the quarter of our patients who have low muscle strength.

\begin{center}\rule{0.5\linewidth}{0.5pt}\end{center}

\pandocbounded{\includegraphics[keepaspectratio]{eso_surgery_files/figure-pdf/pneumonia_fig2b-1.pdf}}

\begin{center}\rule{0.5\linewidth}{0.5pt}\end{center}

\subsection{Muscle Strength and Risk after
Esophagectomy}\label{muscle-strength-and-risk-after-esophagectomy}

The results of our research suggest a simple answer: The risk of
pneumonia is related to a patient's muscle strength.{]}

\pandocbounded{\includegraphics[keepaspectratio]{images/emancipation-156066_1280.png}}

\subsection{}\label{section}

Now this doesn't mean that you need to look like this to prevent
pneumonia after your esophagectomy{]}

\pandocbounded{\includegraphics[keepaspectratio]{images/man-461195_1920.jpg}}

\begin{center}\rule{0.5\linewidth}{0.5pt}\end{center}

The good news is that you can increase your muscle strength before
surgery in two very simple ways:

\begin{itemize}
\item
  Good nutrition with adequate intake of protein
\item
  Exercise
\end{itemize}

\begin{center}\rule{0.5\linewidth}{0.5pt}\end{center}

\subsection{Good News}\label{good-news}

with proper nutrition and exercise, you can increase your muscle
strength, and we have good reason to believe this will reduce your risk
of complications after esophagectomy.

When you meet with your surgery team, be sure to ask them about pain
control after surgery and how you can increase your muscle strength.

\begin{center}\rule{0.5\linewidth}{0.5pt}\end{center}

In the next video in our series, you will learn about how to choose a
hospital and a surgeon for esophageal surgery:

\href{32-Surgeon.html}{Choosing a Hospital and Surgeon for
Esophagectomy}

\begin{center}\rule{0.5\linewidth}{0.5pt}\end{center}

We hope you have found this video helpful. This videos and others like
it are designed to

educate patients and families about esophageal cancer

and equip them for their discussions with their esophageal cancer care
team.

As always, these videos are no substitute for expert medical advice.

\begin{center}\rule{0.5\linewidth}{0.5pt}\end{center}

Feel free to leave a comment or a question, or if you have suggestions
for future videos.

\begin{center}\rule{0.5\linewidth}{0.5pt}\end{center}




\end{document}
