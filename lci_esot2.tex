% Options for packages loaded elsewhere
% Options for packages loaded elsewhere
\PassOptionsToPackage{unicode}{hyperref}
\PassOptionsToPackage{hyphens}{url}
\PassOptionsToPackage{dvipsnames,svgnames,x11names}{xcolor}
%
\documentclass[
  letterpaper,
  DIV=11,
  numbers=noendperiod]{scrartcl}
\usepackage{xcolor}
\usepackage{amsmath,amssymb}
\setcounter{secnumdepth}{-\maxdimen} % remove section numbering
\usepackage{iftex}
\ifPDFTeX
  \usepackage[T1]{fontenc}
  \usepackage[utf8]{inputenc}
  \usepackage{textcomp} % provide euro and other symbols
\else % if luatex or xetex
  \usepackage{unicode-math} % this also loads fontspec
  \defaultfontfeatures{Scale=MatchLowercase}
  \defaultfontfeatures[\rmfamily]{Ligatures=TeX,Scale=1}
\fi
\usepackage{lmodern}
\ifPDFTeX\else
  % xetex/luatex font selection
\fi
% Use upquote if available, for straight quotes in verbatim environments
\IfFileExists{upquote.sty}{\usepackage{upquote}}{}
\IfFileExists{microtype.sty}{% use microtype if available
  \usepackage[]{microtype}
  \UseMicrotypeSet[protrusion]{basicmath} % disable protrusion for tt fonts
}{}
\makeatletter
\@ifundefined{KOMAClassName}{% if non-KOMA class
  \IfFileExists{parskip.sty}{%
    \usepackage{parskip}
  }{% else
    \setlength{\parindent}{0pt}
    \setlength{\parskip}{6pt plus 2pt minus 1pt}}
}{% if KOMA class
  \KOMAoptions{parskip=half}}
\makeatother
% Make \paragraph and \subparagraph free-standing
\makeatletter
\ifx\paragraph\undefined\else
  \let\oldparagraph\paragraph
  \renewcommand{\paragraph}{
    \@ifstar
      \xxxParagraphStar
      \xxxParagraphNoStar
  }
  \newcommand{\xxxParagraphStar}[1]{\oldparagraph*{#1}\mbox{}}
  \newcommand{\xxxParagraphNoStar}[1]{\oldparagraph{#1}\mbox{}}
\fi
\ifx\subparagraph\undefined\else
  \let\oldsubparagraph\subparagraph
  \renewcommand{\subparagraph}{
    \@ifstar
      \xxxSubParagraphStar
      \xxxSubParagraphNoStar
  }
  \newcommand{\xxxSubParagraphStar}[1]{\oldsubparagraph*{#1}\mbox{}}
  \newcommand{\xxxSubParagraphNoStar}[1]{\oldsubparagraph{#1}\mbox{}}
\fi
\makeatother


\usepackage{longtable,booktabs,array}
\usepackage{calc} % for calculating minipage widths
% Correct order of tables after \paragraph or \subparagraph
\usepackage{etoolbox}
\makeatletter
\patchcmd\longtable{\par}{\if@noskipsec\mbox{}\fi\par}{}{}
\makeatother
% Allow footnotes in longtable head/foot
\IfFileExists{footnotehyper.sty}{\usepackage{footnotehyper}}{\usepackage{footnote}}
\makesavenoteenv{longtable}
\usepackage{graphicx}
\makeatletter
\newsavebox\pandoc@box
\newcommand*\pandocbounded[1]{% scales image to fit in text height/width
  \sbox\pandoc@box{#1}%
  \Gscale@div\@tempa{\textheight}{\dimexpr\ht\pandoc@box+\dp\pandoc@box\relax}%
  \Gscale@div\@tempb{\linewidth}{\wd\pandoc@box}%
  \ifdim\@tempb\p@<\@tempa\p@\let\@tempa\@tempb\fi% select the smaller of both
  \ifdim\@tempa\p@<\p@\scalebox{\@tempa}{\usebox\pandoc@box}%
  \else\usebox{\pandoc@box}%
  \fi%
}
% Set default figure placement to htbp
\def\fps@figure{htbp}
\makeatother





\setlength{\emergencystretch}{3em} % prevent overfull lines

\providecommand{\tightlist}{%
  \setlength{\itemsep}{0pt}\setlength{\parskip}{0pt}}



 


\KOMAoption{captions}{tableheading}
\makeatletter
\@ifpackageloaded{caption}{}{\usepackage{caption}}
\AtBeginDocument{%
\ifdefined\contentsname
  \renewcommand*\contentsname{Table of contents}
\else
  \newcommand\contentsname{Table of contents}
\fi
\ifdefined\listfigurename
  \renewcommand*\listfigurename{List of Figures}
\else
  \newcommand\listfigurename{List of Figures}
\fi
\ifdefined\listtablename
  \renewcommand*\listtablename{List of Tables}
\else
  \newcommand\listtablename{List of Tables}
\fi
\ifdefined\figurename
  \renewcommand*\figurename{Figure}
\else
  \newcommand\figurename{Figure}
\fi
\ifdefined\tablename
  \renewcommand*\tablename{Table}
\else
  \newcommand\tablename{Table}
\fi
}
\@ifpackageloaded{float}{}{\usepackage{float}}
\floatstyle{ruled}
\@ifundefined{c@chapter}{\newfloat{codelisting}{h}{lop}}{\newfloat{codelisting}{h}{lop}[chapter]}
\floatname{codelisting}{Listing}
\newcommand*\listoflistings{\listof{codelisting}{List of Listings}}
\makeatother
\makeatletter
\makeatother
\makeatletter
\@ifpackageloaded{caption}{}{\usepackage{caption}}
\@ifpackageloaded{subcaption}{}{\usepackage{subcaption}}
\makeatother
\usepackage{bookmark}
\IfFileExists{xurl.sty}{\usepackage{xurl}}{} % add URL line breaks if available
\urlstyle{same}
\hypersetup{
  pdftitle={Early-Stage Cancer of the Esophagus and GE Junction},
  colorlinks=true,
  linkcolor={blue},
  filecolor={Maroon},
  citecolor={Blue},
  urlcolor={Blue},
  pdfcreator={LaTeX via pandoc}}


\title{Early-Stage Cancer of the Esophagus and GE Junction}
\author{}
\date{}
\begin{document}
\maketitle


\subsection{Anatomy}\label{anatomy}

Food moves from the throat

\(\rightarrow\) esophagus

\(\rightarrow\) stomach

\(\rightarrow\) small bowel (jejunum)

\pandocbounded{\includegraphics[keepaspectratio]{images/Eso_Anatomy_Labels.png}}{]}

\subsection{Types of Esophageal
Cancer}\label{types-of-esophageal-cancer}

There are two common types of esophageal cancer

\begin{itemize}
\tightlist
\item
  Adenocarcinoma
\item
  Squamous Cell Carcinoma
\end{itemize}

In many ways, these to different types of esophageal cancer behave the
same.

We will see later in this video, however, that the treatment
\textbf{can} be different depending upon whether the cancer is
adenocarcinoma or squamous cell carcinoma.

\subsection{Cancer Staging}\label{cancer-staging}

Staging refers to the tests to determine

\begin{itemize}
\tightlist
\item
  How large is the tumor?
\item
  Has there been spread to lymph nodes?
\item
  Has it spread to other parts of the body?
\end{itemize}

\textbf{Treatment options depend upon the cancer stage}

\subsection{Esophageal Cancer Staging}\label{esophageal-cancer-staging}

\begin{itemize}
\tightlist
\item
  \textbf{T} = Tumor - How deep has cancer grown into the wall of the
  esophagus?
\item
  \textbf{N} = Nodes - Has cancer spread to the lymph nodes?
\item
  \textbf{M} = Metastasis - Has the cancer spread to other parts of the
  body? lungs or liver?
\end{itemize}

\subsection{Layers of the Wall of the
Esophagus}\label{layers-of-the-wall-of-the-esophagus}

\begin{itemize}
\tightlist
\item
  Mucosa - Inner layer
\item
  Muscle Wall (muscularis)
\item
  Lymph nodes located in fat outside the muscle
\end{itemize}

\pandocbounded{\includegraphics[keepaspectratio]{images/tumor1_ai.png}}

\subsection{Early Stage Cancers}\label{early-stage-cancers}

Cancers start on the very inside of the layer called the mucosa

\pandocbounded{\includegraphics[keepaspectratio]{images/tumor21_ai.png}}

\subsection{Locally-advanced Cancers}\label{locally-advanced-cancers}

Over time, cancers can grow into the muscular wall

\pandocbounded{\includegraphics[keepaspectratio]{images/tumor24_ai.png}}

\subsection{Lymph Nodes}\label{lymph-nodes}

In some cases, cancer cells can break off from the main tumor and spread
to lymph nodes

\pandocbounded{\includegraphics[keepaspectratio]{images/tumor25b_ai.png}}

\subsection{T Stage}\label{t-stage}

Cancers are categorized based upon the thickness of the tumor, known as
the T stage

\pandocbounded{\includegraphics[keepaspectratio]{images/tumor_t_full.png}}

\subsection{N Stage}\label{n-stage}

Cancers are categorized by whether there is spread to the lymph nodes.

\begin{itemize}
\tightlist
\item
  \textbf{N0} cancers have not spread to the lymph nodes
\item
  \textbf{N1} cancers have spread to the lymph nodes.
\end{itemize}

\pandocbounded{\includegraphics[keepaspectratio]{images/tumor_t3_nodes_labels.png}}

\subsection{M Stage}\label{m-stage}

Some cancers can also spread from the esophagus to the lungs or liver

\begin{itemize}
\tightlist
\item
  \textbf{M0} cancers have not spread to other parts of the body
\item
  \textbf{N1} cancers have spread to other parts of the body such as
  lungs or liver
\end{itemize}

\pandocbounded{\includegraphics[keepaspectratio]{images/Eso_M_Stage.png}}

\subsection{PET scan}\label{pet-scan}

A PET scan is similar to a CT scan, and uses a small amount of tracer to
light up areas of cancer.

\pandocbounded{\includegraphics[keepaspectratio]{images/PetImage.png}}

\subsection{Endoscopic Ultrasound}\label{endoscopic-ultrasound}

Endoscopic ultrasound (EUS) is a procedure similar to upper endoscopy
(EGD) which has an ultrasound probe at the bottom of the scope. This
allows measuring the thickness of the cancer. Endoscopic ultrasound can
help determine the T stage of the cancer.

\pandocbounded{\includegraphics[keepaspectratio]{images/EUSArtboard.png}}

\subsection{Laparoscopy}\label{laparoscopy}

Some esophageal cancers can spread inside the abdominal cavity. These
areas of spread can be very small, as small as a grain of rice.

In order to detect spread within the abdominal cavity, a proceduce
called a laparoscopy can be performed in some some patients.

\pandocbounded{\includegraphics[keepaspectratio]{images/Eso_Laparoscopy.png}}

\subsection{Laparoscopy}\label{laparoscopy-1}

A laparoscopy is performed under a general anesthetic.

\begin{itemize}
\tightlist
\item
  Several incisions 1/4'' long
\item
  A telescope is inserted to look inside the abdominal cavity.
\item
  Biopsies can be performed.
\end{itemize}

\pandocbounded{\includegraphics[keepaspectratio]{images/Eso_Laparoscopy.png}}

\subsection{Treatment Plans}\label{treatment-plans}

\begin{itemize}
\tightlist
\item
  Superficial (T1) \(\Rightarrow\) Endoscopic Therapy
\item
  Localized (T1b/T2) \(\Rightarrow\) Surgery
\item
  Locally-advanced (T3/N1) \(\Rightarrow\) Chemo \(\pm\) Radiation
  \(\rightarrow\)Surgery
\item
  Metastatic (M1) \(\Rightarrow\) Chemotherapy
\end{itemize}

\subsection{Localized Cancers}\label{localized-cancers}

Patients with locally-advanced esophageal cancer have smaller tumors
which have not invaded all the way throug the muscle wall and are too
large to remove by endoscopic therapy

\pandocbounded{\includegraphics[keepaspectratio]{images/Eso_tumor_t12t2.png}}

\subsection{Localized Cancers}\label{localized-cancers-1}

Localized cancers rarely spread to the lymph nodes, which means that
surgery alone can often remove the cancer completely.

Localized cancers are less likely to need chemotherapy or radiation
because the cancer is localized to the wall of the esophagus.

\pandocbounded{\includegraphics[keepaspectratio]{images/Eso_tumor_t12t2.png}}

\subsection{Localized Cancers}\label{localized-cancers-2}

Localized cancers are not very common, because most people don't know
that they have esophageal cancer until they have difficulty eating. The
majority of patients who have difficulty eating have a T3 tumor
\emph{and} usually need either chemotherapy and/or radiation to prevent
recurrences

\pandocbounded{\includegraphics[keepaspectratio]{images/Eso_tumor_t12t2.png}}

\subsection{Diagnosis}\label{diagnosis}

Localized cancers are generally diagnosed with a combination of
endoscopic ultrasound and PET scan.

\begin{itemize}
\tightlist
\item
  Endoscopic ultrasound can determine the T stage
\item
  PET scan can look at the nearby lymph nodes
\end{itemize}

\pandocbounded{\includegraphics[keepaspectratio]{images/Eso_tumor_100_1600.png}}

\subsection{Surgery}\label{surgery}

Surgery to remove the esophagus is frequently done for localized
esophageal cancers

\pandocbounded{\includegraphics[keepaspectratio]{images/Eso_IvorLewis_simple.png}}

\subsection{Surgery for Localized Esophageal
Cancer}\label{surgery-for-localized-esophageal-cancer}

If surgery is performed, the cancer in the esophagus is examined by the
pathologist to confirm the precise thickness of the tumor and its T
stage

About 25\% of the time, the pathologist finds that the tumor is actually
T3 or N1, in which case chemotherapy and/or radiation is needed after
surgery.

On the other hand, in 75\% of cases surgery is all the therapy that is
required and there is no need for chemotherapy and/or radiation

\subsection{Chemotherapy + Radiation for Localized Esophageal
Cancer}\label{chemotherapy-radiation-for-localized-esophageal-cancer}

An alternative to surgery is to start with a combination of chemotherapy
and radiation therapy.

We know that in some cases, chemotherapy + radiation can be curative for
esophageal cancer without the need for surgery.

For adenocarcicnoma, about 25\% of cases are cured with chemotherapy +
radiation

For squmaous cell carcinoma, about 40\% of cases are cured with
chemotherapy + radiation

\subsection{Followup after Chemotherapy +
Radiation}\label{followup-after-chemotherapy-radiation}

The challenge here is that it's difficult to know right away whether
chemotherapy + radiation has been effective for esophageal cancer.

In most cases, scans and upper endoscopy performed after chemotherapy +
radiation show no signs of cancer, but only a minority of cases are
actually cured.

In can take up to two years to know with certainty whether or not an
esophageal cancer has been cured by chemotherapy + radiation.

\subsection{Chemotherapy + Radiation CROSS
Trial}\label{chemotherapy-radiation-cross-trial}

\textbf{Chemotherapy + Radiation} \(\rightarrow\)Surgery \(\Rightarrow\)
Longer Survival

Chemotherapy and radiation were given together over six weeks

\subsection{Chemotherapy + Radiation CROSS
Trial}\label{chemotherapy-radiation-cross-trial-1}

A typical schedule for chemotherapy + radiation:

\begin{itemize}
\tightlist
\item
  Chemotherapy once per week for six weeks
\item
  Radiation five days per week for six weeks (28 treatments)
\item
  PET scan 4 weeks after the end of radiation
\item
  EGD every 3-6 months for 2 years
\item
  CT scan every 6-12 months
\end{itemize}

\subsection{Chemotherapy + Radiation - Side
Effects}\label{chemotherapy-radiation---side-effects}

Radiation kills cancer cells, but can also cause irritation of the
lining of the esophagus.

This can make swallowing more challenging the last two weeks of therapy.

A feeding tube is sometimes needed to help with hydration and nutrition.

\subsection{Chemotherapy}\label{chemotherapy}

Chemotherapy drugs are administered intravenously.

There are several options for intravenous access:

\begin{itemize}
\tightlist
\item
  Peripheral IVs in the hand
\item
  PICC line (Peripheral Inserted Central Catheter)
\item
  Central Venous Port
\end{itemize}

\subsection{Peripheral IVs}\label{peripheral-ivs}

Some patients can be treated with an intravenous line placed in the hand
or arm for each dose of chemotherapy. The catheter is placed at the
beginning of each dose and removed that day.

\pandocbounded{\includegraphics[keepaspectratio]{images/access_peripheral-venous-catheter-427167-7_960_720-pixabay.jpeg}}

\subsection{PICC Lines}\label{picc-lines}

A PICC line is placed in Radiology and stays in place during the
treatment course

\pandocbounded{\includegraphics[keepaspectratio]{images/comm_picc.jpeg}}

\subsection{Central Venous Port}\label{central-venous-port}

A central venous port is an implantable device that makes the
administration of chemotherapy easier

\pandocbounded{\includegraphics[keepaspectratio]{images/access_cv_port.png}}

\subsection{Central Venous Port}\label{central-venous-port-1}

A central venous port is typically placed underneath the skin below the
right collarbone

\pandocbounded{\includegraphics[keepaspectratio]{images/access_cv_port_1700.png}}

\subsection{Central Venous Port}\label{central-venous-port-2}

When it is time for chemotherapy, a needle is inserted through the skin
into the port

\pandocbounded{\includegraphics[keepaspectratio]{images/access_cv_port_detail_1700.png}}

\subsection{Restaging}\label{restaging}

CT or PET scan will be performed after preoperative therapy

\subsection{Surgery for Esophageal
Cancer}\label{surgery-for-esophageal-cancer}

Surgery for esophageal cancer is performed for:

\begin{itemize}
\tightlist
\item
  Superficial Tumors (T1) not removed by endoscopy
\item
  Localized Tumors (T2 N0 M0)
\item
  Locally Advanced (T3 M0) after preop therapy
\end{itemize}

\subsection{Goals of Surgery}\label{goals-of-surgery}

\begin{itemize}
\tightlist
\item
  Remove tumor from esophagus
\item
  Remove surrounding lymph nodes
\item
  Create a new esophagus
\end{itemize}

\pandocbounded{\includegraphics[keepaspectratio]{images/Eso_Resection1_ai.png}}

\subsection{Ivor Lewis (Transthoracic)
Esophagectomy}\label{ivor-lewis-transthoracic-esophagectomy}

\begin{itemize}
\tightlist
\item
  Removes tumor and lower 1/3 esophagus
\item
  Removes surrounding lymph nodes
\item
  GI tract reconstructed
\end{itemize}

\pandocbounded{\includegraphics[keepaspectratio]{images/Eso_Resection2_ai.png}}

\subsection{Reconstruction}\label{reconstruction}

New esophagus is created from the stomach in the abdomen by fashioning
it into a tube.

\pandocbounded{\includegraphics[keepaspectratio]{images/Eso_Resection3_ai.png}}

\subsection{Ivor Lewis esophagectomy}\label{ivor-lewis-esophagectomy}

The new esophagus is now brought up into the chest. A connection is made
between the esophagus and the stomach, called an \emph{anastomosis}.

\pandocbounded{\includegraphics[keepaspectratio]{images/Eso_IvorLewis_Anastomosis.png}}

\subsection{Minimally-invasive Ivor
Lewis}\label{minimally-invasive-ivor-lewis}

\begin{itemize}
\tightlist
\item
  Small incisions abdomen and chest
\item
  Surgical telescope and instruments
\item
  Smaller incisions \(\rightarrow\) faster recovery and less discomfort
\end{itemize}

\pandocbounded{\includegraphics[keepaspectratio]{images/MIE_IvorLewisArtboard.png}}

\subsection{Open Ivor Lewis}\label{open-ivor-lewis}

Mininally-invasive approach feasible in 95\% of cases

In some cases, an open approach is still necessary.

\pandocbounded{\includegraphics[keepaspectratio]{images/IvorLewisArtboard.png}}

\subsection{Total Esophagectomy}\label{total-esophagectomy}

For patients with tumors in the upper esophagus, we need to remove more
of the esophagus

We need to remove the whole esophagus, including the portion in the neck

\pandocbounded{\includegraphics[keepaspectratio]{images/Eso_ProxTumorArtboard.png}}

\subsection{McKeown Esophagectomy}\label{mckeown-esophagectomy}

\pandocbounded{\includegraphics[keepaspectratio]{images/Eso_ResectionTotalArtboard.png}}
All of esophagus removed

\pandocbounded{\includegraphics[keepaspectratio]{images/Eso_MIE_McKeownArtboard.png}}
Connection made in the neck

\subsection{Risks of Esophagectomy}\label{risks-of-esophagectomy}

Esophagectomy is a complex operation, with a real risk of complications.

Two significant complications:

\begin{itemize}
\tightlist
\item
  Anastomotic leak
\item
  Pneumonia
\end{itemize}

\subsection{Anastomotic Leak}\label{anastomotic-leak}

The anastomosis is surgical connection between the esophagus and the
stomach.

\pandocbounded{\includegraphics[keepaspectratio]{images/Eso_IvorLewis_Anastomosis.png}}

\subsection{Anastomotic Leak}\label{anastomotic-leak-1}

If healing doesn't occur:

\begin{itemize}
\tightlist
\item
  Leakage of fluid from the esophagus
\item
  Infection in the space between the lungs
\item
  Requires additional time in the hospital
\end{itemize}

\pandocbounded{\includegraphics[keepaspectratio]{images/Eso_IvorLewis_Leak.png}}

\subsection{Anastomotic Leak}\label{anastomotic-leak-2}

If leak occurs:

\begin{itemize}
\tightlist
\item
  Some leaks will seal
\item
  Stent may be required to help healing
\item
  Occasionally additional surgey is required
\end{itemize}

\pandocbounded{\includegraphics[keepaspectratio]{images/Eso_IvorLewis_Leak.png}}

\subsection{Anastomotic Leak}\label{anastomotic-leak-3}

Risk of leak depends on:

\begin{itemize}
\tightlist
\item
  Type of operation performed
\item
  Nutritional status of patient
\item
  Experience of the surgeon
\end{itemize}

\pandocbounded{\includegraphics[keepaspectratio]{images/Eso_IvorLewis_Leak.png}}

\subsection{Pneumonia}\label{pneumonia}

\begin{itemize}
\item
  Occurs in 10-15\% of patients after esophagectomy.
\item
  Requires treatment with antibiotics
\item
  Requires a longer hospitalization.
\end{itemize}

\pandocbounded{\includegraphics[keepaspectratio]{images/Eso_LungsArtboard.png}}

\subsection{Preventing Pneumonia}\label{preventing-pneumonia}

Several ways to help prevent pneumonia:

\begin{itemize}
\tightlist
\item
  Deep breathing
\item
  Coughing
\item
  Walking
\end{itemize}

After surgery, this means:

\begin{itemize}
\tightlist
\item
  Sitting in a chair most of the day
\item
  Walking in the halls as soon as possible
\end{itemize}

\subsection{Minimally-invasive
Esophagectomy}\label{minimally-invasive-esophagectomy}

\pandocbounded{\includegraphics[keepaspectratio]{images/mie_abd.png}}

\pandocbounded{\includegraphics[keepaspectratio]{images/mie_chest.png}}

\subsection{Risks of Surgery}\label{risks-of-surgery}

Risks related to anesthesia

\begin{itemize}
\tightlist
\item
  Heart attack (5\%)
\item
  Irregular heart rhythm (15\%)
\item
  Pneumonia (10\%)
\item
  Blood clots in legs (\textless5\%)
\item
  Pulmonary embolism (2\%)
\end{itemize}

\subsection{Risks of Surgery}\label{risks-of-surgery-1}

Risks related to Surgery

\begin{itemize}
\tightlist
\item
  Anastomotic leak (5\%)
\item
  Stricture at anastomosis (15\%)
\item
  Death within 90 days of surgery

  \begin{itemize}
  \tightlist
  \item
    Low risk patients = 2\%
  \item
    Intermediate risk = 10\%
  \item
    High risk = 30\%
  \end{itemize}
\end{itemize}

\subsection{Risks of Surgery}\label{risks-of-surgery-2}

\begin{longtable}[]{@{}lcc@{}}
\caption{Risks of Death within 90 Days of Surgery}\tabularnewline
\toprule\noalign{}
& Age \textless75 & Age \textgreater75 \\
\midrule\noalign{}
\endfirsthead
\toprule\noalign{}
& Age \textless75 & Age \textgreater75 \\
\midrule\noalign{}
\endhead
\bottomrule\noalign{}
\endlastfoot
Normal Muscle (75\%) & 2\% & 10\% \\
Low Muscle (25\%) & 10\% & 30\% \\
\end{longtable}

\subsection{Day Prior to Surgery}\label{day-prior-to-surgery}

\begin{itemize}
\tightlist
\item
  Clear liquids for 24 hours prior to surgery
\item
  Check with Pre-op nurse regarding medicines day prior to surgery
\item
  No tube feedings the night before surgery
\end{itemize}

\subsection{Day of Surgery}\label{day-of-surgery}

\begin{itemize}
\tightlist
\item
  Arrive at 5am -- nothing to eat or drink after midnight.
\item
  Medicines OK w/ a sip of water
\item
  sip of black coffee but \textbf{no cream}.
\item
  Surgery will be cancelled if you have cream/milk
\item
  Waiting room for family and friends on 5th floor
\end{itemize}

\subsection{Epidural Catheter for Pain
Control}\label{epidural-catheter-for-pain-control}

\begin{itemize}
\tightlist
\item
  Remains in place for 2-5 days
\item
  Dosage can be adjusted as needed
\item
  Can make it more difficult to urinate
\item
  May require foley catheter in bladder
\item
  Foley catheter removed after epidural removed
\end{itemize}

\subsection{Intensive Care Unit (ICU) (2-4
days)}\label{intensive-care-unit-icu-2-4-days}

\begin{itemize}
\tightlist
\item
  Surgical ICU on 11th floor
\item
  NG tube in nose to drain stomach and esophagus
\item
  Catheter in bladder
\item
  Chest tube right chest
\item
  Abdominal drains (usually 2)
\item
  Feeding jejunostomy (usually stays in 8 wks)
\end{itemize}

\subsection{Intensive Care Unit (ICU)}\label{intensive-care-unit-icu}

\begin{itemize}
\tightlist
\item
  Bladder catheter removed → check that bladder empties properly
\item
  Chest tube removed (day 2-4) → follow-up x-ray
\item
  Fluid emptied from drains every few hours
\item
  Start tube feedings by feeding
\item
  Feeding jejunostomy (stays in 8 weeks)
\end{itemize}

\subsection{Ward - 6Tower}\label{ward---6tower}

\begin{itemize}
\tightlist
\item
  Jejunostomy feeds started
\item
  Up in a chair most of the day
\item
  Walking in the halls

  \begin{itemize}
  \tightlist
  \item
    Start with assistance
  \item
    Improves lung function
  \item
    Prevents loss of muscle strength
  \end{itemize}
\end{itemize}

\subsection{Jejunostomy Feeds}\label{jejunostomy-feeds}

Jejunostomy tube feeds

\begin{itemize}
\tightlist
\item
  Start continuous (24 hours)
\item
  Convert to night-time only (16 hours)
\end{itemize}

Water administered through feeding tube

\begin{itemize}
\tightlist
\item
  Usually 8oz 4 times/day
\item
  Important to prevent dehydration
\end{itemize}

\subsection{Jejunostomy Tube}\label{jejunostomy-tube}

\begin{itemize}
\tightlist
\item
  Nutrition to bypasses the esophagus and stomach
\item
  Placed in small intestine
\item
  Pump administers feedings slowly
\item
  Feeding done at night
\end{itemize}

\pandocbounded{\includegraphics[keepaspectratio]{images/nutrition_jtube.png}}

\subsection{Jejunostomy Typical
Regimen}\label{jejunostomy-typical-regimen}

\begin{itemize}
\tightlist
\item
  Jejunostomy tube feeds for 16 hours (6pm-10am)

  \begin{itemize}
  \tightlist
  \item
    Men: 75mL/hour x 16 hours = 5 cartons
  \item
    Women: 60mL/hour x 16 hours = 4 cartons
  \end{itemize}
\item
  Water 240ml (8oz) via syringe 4x/day
\end{itemize}

Hospital nurses will teach use of the feeding tube

\subsection{Jejunostomy Feeds with
Diabetes}\label{jejunostomy-feeds-with-diabetes}

Jejunostomy feedings elevate blood sugars

\begin{itemize}
\tightlist
\item
  Insulin may be required along with feeds
\end{itemize}

Typical Pattern for tube feeds

\begin{itemize}
\tightlist
\item
  Feeds run via pump from 6pm to 10am
\item
  Insulin at 6pm (70/30 insulin)
\item
  Insulin at Midnight (70/30 insulin)
\item
  No insulin if tube feedings are not run
\end{itemize}

\subsection{Jejunostomy Video}\label{jejunostomy-video}

A video is available to help become familiar with the feeding
jejunostomy

\pandocbounded{\includegraphics[keepaspectratio]{images/nutrition_jejunostomy_qrcode.png}}

\subsection{Activity after Surgery}\label{activity-after-surgery}

\begin{itemize}
\tightlist
\item
  Up in chair most of the day
\item
  Walking with help from nurse/Physical Therapist
\item
  Goals:

  \begin{itemize}
  \tightlist
  \item
    Improve lung function
  \item
    Prevent muscle loss
  \end{itemize}
\end{itemize}

\subsection{Nasogastric (NG) Tube}\label{nasogastric-ng-tube}

Tube passed through nose into stomach

\begin{itemize}
\tightlist
\item
  Drains fluid from stomach
\item
  Prevents vomiting
\end{itemize}

Upper GI X-ray on 2nd or 3rd day after surgery

\begin{itemize}
\tightlist
\item
  If stomach empties well \(\rightarrow\) NG tube removed
\item
  Otherwise, X-ray repeated 2-3 days later
\end{itemize}

\subsection{Swallowing Evaluation}\label{swallowing-evaluation}

Once NG tube has been removed:

Modified barium swallow in radiology

\begin{itemize}
\tightlist
\item
  Drink a white chalky liquid (barium)
\item
  Look for proper swallowing function
\item
  70\% of patients \(\Rightarrow\) liquids started by mouth
\end{itemize}

\subsection{Oral Intake at Home}\label{oral-intake-at-home}

Most are taking protein shakes when they go home

Protein shakes are started after tolerating water\\

\begin{itemize}
\tightlist
\item
  2 oz per hour to start
\item
  4 oz per hour if 2oz are tolerated well
\end{itemize}

\subsection{Discharge}\label{discharge}

Goal: ready to leave day \#6/7 after surgery

\begin{itemize}
\tightlist
\item
  Night-time tube feedings (6pm to 10am)
\item
  Nutrition by mouth (70\% of patients)

  \begin{itemize}
  \tightlist
  \item
    1 oz of water per hour by mouth OR
  \item
    Protein shakes 4oz every 2 hours
  \end{itemize}
\item
  Water through tube 8oz four times per day
\item
  Home care nursing (feeding tube teaching)
\item
  Home infusion (tube feeding supplies)
\end{itemize}

\subsection{Nutrition after Surgery}\label{nutrition-after-surgery}

At discharge home:

\begin{itemize}
\tightlist
\item
  Protein shakes 4oz every 2 hrs
\item
  Tube feeds 4-5 cans at night (6pm-10am)
\end{itemize}

10-12 Days: Increase protein shakes

\begin{itemize}
\tightlist
\item
  Tube feeds 3-4 cans at night
\end{itemize}

Three weeks: Post-esophagectomy Diet

8-12 weeks: Remove feeding tube (in office)

\subsection{\texorpdfstring{Transition from Tube Feeds \(\rightarrow\)
Eating}{Transition from Tube Feeds \textbackslash rightarrow Eating}}\label{transition-from-tube-feeds-rightarrow-eating}

Dietitian will calculate daily protein goal

\begin{itemize}
\tightlist
\item
  Typically 60-75 grams protein/day
\item
  Each carton of tube feeding has 15 grams

  \begin{itemize}
  \tightlist
  \item
    75 grams protein = 5 cartons/night
  \end{itemize}
\item
  More intake by mouth \(\rightarrow\) tube feeds reduced
\end{itemize}

Spread out protein during the day (20gm/meal)

\begin{itemize}
\tightlist
\item
  Three meals + 2-3 high-protein snacks
\end{itemize}

\subsection{Post-esophagectomy Diet}\label{post-esophagectomy-diet}

\begin{itemize}
\tightlist
\item
  Soft Consistency
\item
  High Protein
\item
  Avoid sugary liquids (can cause `dumping')
\item
  Avoid raw vegetables (and salads)
\item
  Eating

  \begin{itemize}
  \tightlist
  \item
    Small, frequent meals
  \item
    Sit up for 30-45 minutes after eating
  \item
    Avoid eating within 2 hours of bedtime
  \end{itemize}
\end{itemize}

\subsection{Medicines at Home - Pain}\label{medicines-at-home---pain}

Acetaminophen (Tylenol) 1000mg 4x/day

Gabapentin 300mg 3 times/day

Oxycodone

\begin{itemize}
\tightlist
\item
  As needed in addition to Tylenol/gabapentin
\item
  Will begin reducing dose at first postop visit
\item
  Can usually discontinue by 4 weeks
\item
  NO DRIVING WHILE ON OXYCODONE
\end{itemize}

\subsection{Non-steroidals Anti Inflammatory
(NSAID)}\label{non-steroidals-anti-inflammatory-nsaid}

Non-steroidal anti-inflammatories (Celebrex)

\begin{itemize}
\tightlist
\item
  200 mg every 12 hours starting at 2 weeks
\end{itemize}

NO GOODY POWDERS OR BCs

\begin{itemize}
\tightlist
\item
  (Can cause permanent scarring at the surgery site)
\end{itemize}

\subsection{Acid Blockers = Proton Pump
Inhibitors}\label{acid-blockers-proton-pump-inhibitors}

Examples include ompeprazole and pantoprazole

\begin{itemize}
\tightlist
\item
  Will stay on for at 1-2 years to prevent acid reflux
\item
  Important in preventing scarring at anastomosis (new connection
  between esophagus and stomach)
\item
  To administer through feeding tube, open capsule and resuspend beads
  in 60mL (2oz) of water
\end{itemize}

\subsection{Medicines at Home}\label{medicines-at-home}

Reglan -- Helps stomach empty

\begin{itemize}
\tightlist
\item
  Will plan to stop after six weeks
\item
  0.1\% risk of tardive dyskinesia (nervous tic)
\end{itemize}

Remeron -- Helps improve appetite

\begin{itemize}
\tightlist
\item
  Can cause vivid dreams
\item
  Used for several weeks after surgery
\item
  Will stop within first three months of surgery
\end{itemize}

\subsection{Metoprolol = Beta Blockers}\label{metoprolol-beta-blockers}

\begin{itemize}
\tightlist
\item
  Slows heart rate and lowers blood pressure\\
\item
  Used to prevent rapid heart rate
\item
  Patients not taking a beta blocker prior to surgery \(\rightarrow\)
  wean after after surgery
\item
  Patients taking a beta blockerprior to surgery \(\rightarrow\) return
  to prior dose and drug after surgery
\end{itemize}

\subsection{Sleeping at Home}\label{sleeping-at-home}

Reflux can occur the first few weeks/months after surgery

This improves over the first few months

A wedge pillow can be helpful for sleep

\pandocbounded{\includegraphics[keepaspectratio]{images/wedge_pillow_comm.jpg}}

\subsection{Postoperative Visit at 7-10
Days}\label{postoperative-visit-at-7-10-days}

Check surgical site

\begin{itemize}
\tightlist
\item
  Remove staples (if needed)
\end{itemize}

Adjust medicines as needed

\begin{itemize}
\tightlist
\item
  Insulin (for diabetic patients on insulin)
\item
  Reduce beta blocker medicines
\end{itemize}

Advance diet

Reduce tube feeds

\subsection{After surgery}\label{after-surgery}

Wean off medicines added after surgery

\begin{itemize}
\tightlist
\item
  Pain medicines
\item
  Beta-blockers
\item
  Reglan and Remeron
\end{itemize}

Continue acid blockers for at least 1 year

\subsection{Jejunostomy Removal}\label{jejunostomy-removal}

Jejunostomy tube is removed in the office once you can take in enough
nutrients by mouth

Removal usually around 8 weeks after surgery

May take 30 minutes and some local anesthetic to loosen up the tube for
removal.

\subsection{Nutritional Monitoring after
Surgery}\label{nutritional-monitoring-after-surgery}

You may have difficulty absorbing some nutrients:

\begin{itemize}
\tightlist
\item
  Iron
\item
  Vitamin B12
\item
  Vitamin D
\end{itemize}

\subsection{Nutritional Monitoring after
Surgery}\label{nutritional-monitoring-after-surgery-1}

About 3 months after the jejunostomy tube is removed, we will check
blood levels:

\begin{itemize}
\tightlist
\item
  Iron (ferritin)
\item
  Vitamin B12
\item
  Vitamin D
\end{itemize}

\subsection{Nutritional Replacements after
Surgery}\label{nutritional-replacements-after-surgery}

Vitamin or iron replacements can be ordered by:

\begin{itemize}
\tightlist
\item
  Primary Care Provider (PCP)
\item
  Medical Oncologist
\item
  Surgeon
\end{itemize}

If levels are low

\begin{itemize}
\tightlist
\item
  Replacement
\item
  Repeat testing in 3-6 months
\end{itemize}

\subsection{Team Members - Physicians}\label{team-members---physicians}

Primary Care Provider

Gastroenterologist

Medical Oncologist (chemotherapy)

Radiation Oncologist (radiation)

Surgeons

\begin{itemize}
\tightlist
\item
  Jonathan Salo
\item
  Jeffrey Hagen
\item
  Michael Roach
\end{itemize}

\subsection{Team Members - Support
Staff}\label{team-members---support-staff}

Dietitian - Liz Koch

Nurses

\begin{itemize}
\tightlist
\item
  Brandon Galloway
\item
  Rebecca Wicks
\end{itemize}

Navigator - Laura Swift




\end{document}
